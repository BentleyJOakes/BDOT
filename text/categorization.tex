\section{Categorization}

While sourcing optimizations, we began with a preliminary categorization of them. 

\subsection{Intent}

Some optimizations may attempt to increase the program's performance in some fashion. However, as the model transformation can change the model for the modeller to then further refine, we realized that there are a number of other transformations to annotate the model or change the user's view of the model.

For instance, we discovered a 'style optimization' to change a model structure from a data-flow orientation design to a control-flow orientated design.  This may allow the modeller to model in their own style, potentially reducing errors from unfamiliar structures.

Another 'feedback optimization' is to analyze the model and annotate blocks that perform an errorneous operation such as a divide by 0. This gives the user visual feedback directly on the model as to where error-handling techniques should be performed.

\subsection{Layer}
Some optimizations may be done in a totally platform-independent way. However, some may depend on the target architecture or language.

\subsubsection{Model}
The model optimizations attempt to optimize the model to increase the performance of the model, irregardless of the eventual code target. These optimizations include such things as removing useless blocks or reducing expensive operations.

\subsubsection{Platform-independent}
\subsubsection{Platform-dependent}
